%%%%%%%%%%%%%%%%%%%%%%%%%%%%%%%%%%%%%%%%%%%%%%%%%%%%%%%%%%%%%%%%%% \iffalse %%%%
%                                                                              %
%  Copyright (c) 2014 - Michiel Helvensteijn   (www.mhelvens.net)              %
%                                                                              %
%  https://github.com/mhelvens/latex-pkgloader                                 %
%                                                                              %
%  This work may be distributed and/or modified under the conditions           %
%  of the LaTeX Project Public License, either version 1.3 of this             %
%  license or (at your option) any later version. The latest version           %
%  of this license is in       http://www.latex-project.org/lppl.txt           %
%  and version 1.3 or later is part of all distributions of LaTeX              %
%  version 2005/12/01 or later.                                                %
%                                                                              %
%  This work has the LPPL maintenance status `maintained'.                     %
%                                                                              %
%  The Current Maintainer of this work is Michiel Helvensteijn.                %
%                                                                              %
%%%%%%%%%%%%%%%%%%%%%%%%%%%%%%%%%%%%%%%%%%%%%%%%%%%%%%%%%%%%%%%%%%%%%%% \fi %%%%

\documentclass[a4paper]{pkgloader-packagedoc}

%%%%%%%%%%%%%%%%%%%%%%%%%%%%%%%%%%%%%%%%%%%%%%%%%%%%%%%%%%%%%%%%%%%%%%%%%%%%%%%%
% Setup                                                                        %
%%%%%%%%%%%%%%%%%%%%%%%%%%%%%%%%%%%%%%%%%%%%%%%%%%%%%%%%%%%%%%%%%%%%%%%%%%%%%%%%



%%%%%%%%%%%%%%%%%%%%%%%%%%%%%%%%%%%%%%%%%%%%%%%%%%%%%%%%%%%%%%%%%%%%%%%%%%%%%%%%
% Global Changes                                                               %
%%%%%%%%%%%%%%%%%%%%%%%%%%%%%%%%%%%%%%%%%%%%%%%%%%%%%%%%%%%%%%%%%%%%%%%%%%%%%%%%

\changes{0.1.0}{2014/01/29}
  {initial version}

%%%%%%%%%%%%%%%%%%%%%%%%%%%%%%%%%%%%%%%%%%%%%%%%%%%%%%%%%%%%%%%%%%%%%%%%%%%%%%%%
\begin{document}                                                               %
%%%%%%%%%%%%%%%%%%%%%%%%%%%%%%%%%%%%%%%%%%%%%%%%%%%%%%%%%%%%%%%%%%%%%%%%%%%%%%%%

\maketitle

\section {Introduction}  % % % % % % % % % % % % % % % % % % % % % % % % % % % %

\LaTeX\ can be extended by loading packages, which can add definitions,
change existing ones, or otherwise extend functionality of the language,
using a "\usepackage", "\RequirePackage" or "\RequirePackageWithOptions" command.

While the Turing-complete power of the \TeX\ family is quite useful at times, it does
make it all too easy for independent package authors to step on each others
toes. CTAN is full of conceptually independent packages that cannot be
loaded together, or break if they are not loaded in a specific order.

Yet, until now, there has been no automated package management to speak of.
Document authors are usually told to avoid certain package combinations,
or to load packages in some specific order.
Occasionally, package authors patch their code to be aware of specific
other packages, circumventing known conflicts.
But this makes maintenance more difficult, because these package authors
are `mixing concerns'; they put code related to other packages into their
own package. And it is all done in an ad-hoc fashion.

Enter "pkgloader".
If this package becomes well-used, package authors will be able to
develop in a more modular fashion. And more importantly, some frustration
is taken away for the average document author.



\subsection {Package Description} % % % % % % % % % % % % % % % % % % % % % %

The idea is to load "pkgloader" before loading any other package
% or document class
.
It can then intercept all package
% and class
loading requests, analyze them
and load them properly, taking this burden off the shoulders of
the average document author.

The main file for a \LaTeX\ document using "pkgloader"
looks something like this:
\begin{latex}
\documentclass{article}
\RequirePackage{pkgloader}
    \usepackage{xltxtra}
    \usepackage{graphicx}
     $\vdots$
\LoadPackagesNow

\begin{document}
     $\vdots$
\end{document}
\end{latex}

Between the second and sixth lines, the loading of all packages
%and document classes
is postponed. The "\LoadPackagesNow"
command processes the information and
loads the packages in some valid order. During this process,
`conflict resolving' code may also be run, meant specifically to make
other packages work together properly. If the above code were
compiled without "pkgloader" (using the \XeTeX\ engine, in this case),
the given order between "xltxtra" and "graphicx" would cause an error.
%%
The main advantage to this approach is that the complexity of dealing
with package conflicts is moved to the "pkgloader" package and handled
in a systematic manner.

If you are a document author, this may be all you need to know to use
"pkgloader". If you are interested in more advanced functionality,
read on!



\subsection {The \texttt{\textbackslash Load} Command} % % % % % % % %

Each invocation if the "\Load" command sets up a rule about
a package or packages, which are not necessarily ever loaded.
These rules can come from any number of different sources.
A central registry will be maintained together with "pkgloader" itself,
specifying well-known conflicts and resolutions.
Individual package authors, however, can supply their own rules,
as can document authors. Though ideally, for the
average document author, things should `just work'.

\def\nt#1{\ensuremath{\hbox{\footnotesize$\langle$}\mathit{#1}\hbox{\footnotesize$\rangle$}}}
\def\|{\ \mid\ }

The "\Load" command expects the following syntax:

\vskip\abovedisplayskip
\begingroup
\centering
\begin{tabular}{r@{\ \ \ \ }r@{\ \ \ \ }l}
	\nt{load}           &$::=$&   "\Load"$\ \nt{package}\ \;\nt{clause}_1\ \ldots\ \nt{clause}_i$  \\
	                    &$ \|$&   "\Load error" $\ \ \ \ \,\nt{clause}_1\ \ldots\ \nt{clause}_i$                                              \\[2mm]
	
	\nt{package}        &$::=$&   "["$\;\nt{options}\;$"] {"$\;\nt{pkg}\;$"} ["$\;\nt{version}\;$"]"            \\[2mm]
	
	\nt{clause}         &$::=$&   $\nt{order}\ \ \|\ \ \nt{condition}\ \ \|\ \ \nt{reason}$  \\[2mm]
	
	\nt{order}          &$::=$&   "before {" $\nt{pkg}_1\,$", "$\ldots\,$", "$\nt{pkg}_j$ "}"  \\
		                &$ \|$&   "after  {" $\nt{pkg}_1\,$", "$\ldots\,$", "$\nt{pkg}_j$ "}"  \\
		                &$ \|$&   "early"                                               \\
		                &$ \|$&   "late"                                                \\[2mm]
		
	\nt{condition}      &$::=$&   "always" $\ \ \|\ \ $ "if loaded" $\ \ \|\ \ $ "if {" $\nt{\phi}$ "}"                          \\[2mm]
	
	\nt{\phi}           &$::=$&   $\nt{pkg}$                         $\ \ \|\ \ $
		                          $\nt{\phi}\ $"&&"$\ \nt{\phi}$    $\ \ \|\ \ $
		                          $\nt{\phi}\ $"||"$\ \nt{\phi}$    $\ \ \|\ \ $
		                          "!"$\;\nt{\phi}$                      $\ \ \|\ \ $
		                          "("$\;\nt{\phi}\;$")"                          \\[2mm]
	
	\nt{reason}         &$::=$&   "because {" $\nt{token{-}list}$ "}"    
\end{tabular}
\endgroup
\vskip\belowdisplayskip

\noindent
\nt{pkg} represents a package name.
\nt{token{-}list} should expand to a human-readable text without formatting.
It is used in error messages and such.

The \( \nt{condition} \) clause determines under which package loading
conditions any and all parts of a rule are invoked.
Here is an example of the use of the \( \nt{condition} \) clause:
\begin{latex}
\Load {res-ie-lst} when {inputenc && listings}
\Load {fixltx2e}   always
\end{latex}
|res-ie-lst| (a fictional package built specifically to resolve the
conflict between |inputenc| and |listings|) will be loaded if requested
specifically, or if both |inputenc| and |listings| are loaded. The |fixltx2e| package is
always loaded, as it was created to smooth over some mistakes in the
\LaTeXe\ core.
The |always| keyword makes a rule unconditional. The |if loaded|
directive makes a rule conditional on its package already being loaded anyway.
This can be used to order two packages only when they are being loaded by other
means.

The |before| and |after| keywords should be pretty straightforward. They
can be used for things like:
\begin{latex}
\Load {xltxtra} after {graphicx}
\end{latex}
But the set of \LaTeX\ packages is constantly growing, and it appears that
some big packages should almost always be loaded early in the process,
and others should almost always be loaded late. Therefore the |early|
and |late| stages are provided as a fallback mechanism. If two packages
are not related by an explicit application order, their loading order
may still be decided by their relative stages: |early| before `normal' before |late|.
That way, conflicts may be avoided in a majority of cases.
This is implemented with |pkgloader-early| and |pkgloader-late| stubs in
the loading order graph.

Here is the example from above using
the \( \nt{order} \) clause in addition to the \( \nt{condition} \) clause:
\begin{latex}
\Load {res-ie-lst} when {inputenc && listings}
                  after {inputenc,   listings}
\Load {fixltx2e} always early
\end{latex}
An important observation about the loading order is that it
might form a cycle when contradictory ordering rules are specified:
\begin{latex}
\Load {pkg1} after {pkg2}
\Load {pkg2} after {pkg1}
\end{latex}
In practice this could happen if the authors of |pkg1| and |pkg2|
independently discover a conflict, and both try to solve it by patching
their code and having their own package be loaded last. |pkgloader|
can provide a clear error message when this happens, allowing the
two package authors to seek contact and collaborate on a solution.

Now, about the |error| keyword.
Initially all package combinations are valid. But if two packages
are irredeemably incompatible, their combination can be made to trigger
an error message by a command such as the following:
\begin{latex}
\Load error when {algorithms && pseudocode}
\end{latex}
These two packages provide almost identical functionality and conflict
on many command-names. It was generally agreed upon that they should
never be loaded together. Document authors should simply choose one
or the other.

Finally, the \( \nt{reason} \) clause can be used to supply a
human-readable explanation of a rule.
We finish the above examples by providing reasons:
\begin{latex}[basewidth=.598em]
\Load {res-ie-lst} when {inputenc && listings}
                  after {inputenc,   listings}
     because {it allows the use of 1 byte unicode
              characters in code listings}

\Load {fixltx2e} always early
     because {it fixes some imperfections in LaTeX2e}

\Load error when {algorithms && pseudocode}
     because {they provide almost identical functionality
              and conflict on many command names}
\end{latex}
In the future, reasons will be extracted automatically to
generate documentation. For now, they are displayed with error
messages related to the rule in question.



%%%%%%%%%%%%%%%%%%%%%%%%%%%%%%%%%%%%%%%%%%%%%%%%%%%%%%%%%%%%%%%%%%%%%%%%%%%%%%%%
\end{document}                                                                 %
%%%%%%%%%%%%%%%%%%%%%%%%%%%%%%%%%%%%%%%%%%%%%%%%%%%%%%%%%%%%%%%%%%%%%%%%%%%%%%%%
